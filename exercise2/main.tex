\documentclass{homework}
\usepackage[utf8]{inputenc}
\usepackage{amsmath}
\usepackage{amssymb}
\usepackage{amsthm}
\usepackage{braket}
\usepackage{amsfonts}
\usepackage{mathrsfs}
\usepackage{verbatim}


% CHANGE THE FOLLOW THREE LINES!
\newcommand{\hwname}{Lu Si}
\newcommand{\hwemail}{sil18@mails.tsinghua.edu.cn}
\newcommand{\hwnum}{2}

% CHANGE THESE ONLY ONCE PER CLASS
\newcommand{\hwtype}{Exercise}
\newcommand{\hwclass}{Probability Theory}

\begin{document}

\maketitle
\textbf{Note}: Solving steps are required for all questions.

\question %No.1
The sample space is the set $\Omega=\{1,2,3,4,5,6\}^{3}$. There are $6^3=216$ cases in total. \\
Let us define $A=$\{3 numbers in strictly increasing order\}, 
$B=$\{3 numbers in strictly decreasing order\}. \\
$A=\{123,124,125,126,134,135,136,145,146,156,234,235,236,245,246,256,345,346,356,456\}$,\\
$B=\{321,421,521,621,431,531,631,541,641,651,432,532,632,542,642,652,543,643,653,654\}$.\\
$\mathbb{P}(A)=\mathbb{P}(B)=\frac{20}{216}=\frac{5}{54}$.

\question %No.2
(i) This problem can be written as:\\
\begin{equation}
    \begin{aligned}
        \max_{p_{k}} \quad & -\sum_{k=1}^{n}p_{k}log(p_{k}) \\
        \textrm{s.t.} \quad & \sum_{k=1}^{n}p_{k}=1\\
        &\sum_{k=1}^{n}p_{k}x_{k}=\mu
    \end{aligned}
\end{equation}
We form a Langrangian and introduce Lagrange multipliers $\lambda_{0}$ and $\lambda_{1}$.
\begin{equation}
    \mathcal{L}(p_{k}, \lambda_{0}, \lambda_{1})=-\sum_{k=1}^{n}p_{k}log(p_{k})+
    \lambda_{0}(\sum_{k=1}^{n}p_{k}-1)+
    \lambda_{1}(\sum_{k=1}^{n}p_{k}x_{k}-\mu)
\end{equation}
Take all partial derivatives of $\mathcal{L}$ with respect to $p_{k}, \lambda_{0}$ and $\lambda_{1}$, we get
\begin{equation}
    \begin{aligned}
        &\frac{\partial \mathcal{L}}{\partial p_{k}}=-logp_{k}-1+\lambda_{0}+\lambda_{1}x_{k}=0\\
        &\frac{\partial \mathcal{L}}{\partial \lambda_{0}}=\sum_{k=1}^{n}p_{k}-1=0\\
        &\frac{\partial \mathcal{L}}{\partial \lambda_{1}}=\sum_{k=1}^{n}p_{k}x_{k}-\mu=0
    \end{aligned}
\end{equation}
So, $p_{k}=Cr^{x_{k}}$, where $C=e^{\lambda_{0}-1}, r=e^{\lambda{1}}$.

(ii) Use the conditions that
\begin{equation}
    \begin{aligned}
        &\sum_{k=1}^{\infty}p_{k}=\sum_{k=1}^{\infty}Cr^{k}=1 \\
        &\sum_{k=1}^{\infty}p_{k}x_{k}=\sum_{k=1}^{\infty}Cr^{k}k=\mu
    \end{aligned}
\end{equation}
We get $C=\frac{1}{\mu-1}$ and $r=\frac{\mu-1}{u}$, $p_{k}=(1-\frac{1}{\mu})^{k-1}\frac{1}{u}$ which is geometric distribution with parameter $\frac{1}{\mu}$.

\question %No.3

(i) $\sum_{n=1}^{\infty}\frac{(-1)^{n+1}}{n} =\ln(1-(-1))=\ln 2$ (Taylor series of $\ln(1-x)$).\\
\\
(ii) $\sum_{n=1}^{\infty}(\frac{1}{2n-1}-\frac{1}{4n-2}-\frac{1}{4n})
=\sum_{n=1}^{\infty}\frac{1}{2}(\frac{1}{2n-1}-\frac{1}{2n})=\frac{1}{2}\sum_{n=1}^{\infty}\frac{(-1)^{n+1}}{n}=\frac{ln2}{2}$.\\
\\
(iii) $\sum_{n=1}^{\infty}\frac{1}{n}=1+\frac{1}{2}+\frac{1}{3}+\frac{1}{4}+ \cdots 
> 1+\frac{1}{2}+(\frac{1}{4}+\frac{1}{4})+(\frac{1}{8}+\frac{1}{8}+\frac{1}{8}+\frac{1}{8})+\cdots
=1+\frac{1}{2}+\frac{1}{2}+\frac{1}{2}+\cdots = \infty$.

\question%No.4
(i) $X \sim$ Geometric($p$), we get $\mathbb{E}[X]=\frac{1}{p}, var(X)=\frac{1-p}{p^2}$ 
and $\mathbb{E}(X^2)=(\mathbb{E}(X))^2+var(X)=\frac{2-p}{p^2}$. Then, \\
\begin{equation}
    \begin{aligned}
        \mathbb{E}[X^3 \mid X>1] &=\mathbb{E}[(X-1+1)^3 \mid X>1] \\
                                 &=\mathbb{E}[(X-1)^3+3(X-1)^2+3(X-1)+1 \mid X>1] \\
                                 &=\mathbb{E}[X^3]+3\mathbb{E}[X^2]+3\mathbb{E}[X]+1
    \end{aligned}
\end{equation}
The last equality holds since $P(X-1=k \mid X>1=P(X=k)$.\\
Since $\mathbb{E}[X^3]= \mathbb{E}[X^3 \mid X>1](1-p)+p = 
(\mathbb{E}[X^3]+3\mathbb{E}[X^2]+3\mathbb{E}[X]+1)(1-p)+p$, 
we can get $\mathbb{E}[X^3]=\frac{p^2-6p+6}{p^3}$.\\
\\
Similarly, 
\begin{equation}
    \begin{aligned}
        \mathbb{E}[X^4 \mid X>1] &= \mathbb{E}[(X-1)^4+4(X-1)^3+6(X-1)^2+4(X-1)+1 \mid X>1]\\
                                 &=\mathbb{E}[X^4]+4\mathbb{E}[X^3]+6\mathbb{E}[X^2]+4\mathbb{E}[X]+1
    \end{aligned}
\end{equation}
Since $\mathbb{E}[X^4]= \mathbb{E}[X^4 \mid X>1](1-p)+p$, we can get $\mathbb{E}[X^4]=\frac{-p^3+14p^2-36p+24}{p^4}$.\\
\\
(ii) $X \sim$ Poisson($\lambda$), $\mathbb{E}[X]=\lambda, \mathbb{E}[X^2]=\mathbb{E}[X(X-1)]+\mathbb{E}[X]=\lambda^2+\lambda$.\\
Similarly, we can get $\mathbb{E}[X^3]=\mathbb{E}[X(X-1)(X-2)]+3\mathbb{E}[X^2]-2\mathbb{E}[X]=\lambda^3+3\lambda^2+\lambda$ \\
and $\mathbb{E}[X^4]=\mathbb{E}[X(X-1)(X-2)(X-3)]+6\mathbb{E}[X^3]-11\mathbb{E}[X^2]+6\mathbb{E}[X]=\lambda^4+6\lambda^3+7\lambda^2+\lambda$.


\question %No.5
The sample space is the set $\Omega=\{1,2,\cdots,n\}^k$.  $\mid \Omega \mid = n^k$.\\
Let $A=$\{There exist two people picking the same number\}.\\
$k=\lceil n^\beta \rceil$, when $\beta \ge 1, \mathbb{P}(A)=1$. When $\beta \le 0, \mathbb{P}(A)=0$.\\
When $0<\beta<1$, The number of ways to uniquely select $k$ numbers is $n(n-1)\cdots(n-k+1)$. 
In this case, $\mathbb{P}(A)=1-\frac{n(n-1)\cdots(n-k+1)}{n^k}$.\\
$\lim_{n \to \infty}\mathbb{P}(A)=1-\lim_{n \to \infty}\frac{n}{n}\frac{n-1}{n}\cdots\frac{n-k-1}{n}
=1-\lim_{n \to \infty}(1-\frac{1}{n})(1-\frac{2}{n})\cdots(1-\frac{k-1}{n})
=1-\lim_{n \to \infty}\prod_{i=0}^{k-1}e^{-\frac{i}{n}}\\
=1-\lim_{n \to \infty}e^{-\frac{k^2}{2n}}e^{\frac{k}{2n}}=1-\lim_{n \to \infty}e^{-\frac{k^2}{2n}}$.\\
If $\beta_{0}>\frac{1}{2}, \lim_{n \to \infty}e^{-\frac{k^2}{2n}}=0$. In this case, $\mathbb{P}(A)=1$ when $n \to \infty$.\\ 
If $\beta_{0}<\frac{1}{2}, \lim_{n \to \infty}e^{-\frac{k^2}{2n}}=1$. In this case, $\mathbb{P}(A)=0$ when $n \to \infty$.

\question %No.6
(i) Let $f(m)=\mathbb{P}(\exists n, \textrm{s.t.} S_{n}=m)$, for convience, we set $f(0)=1$.\\
$f(m)=\sum_{i=1}^{\min\{m,6\}}\frac{1}{6}f(m-i)$. 
The python program is as following:
    \begin{verbatim}
        def f(n):
            seq = [1]
            for i in range(n): seq.append(sum(seq[-6:])/6)
            return seq[-1]
    \end{verbatim}
We can get $f(2020)=0.28571428571428575$. \\       
\\
(ii) When $m>6$, the probability of the complement of the event $\{\exists n, \textrm{s.t.} S_{n}=m\}$ 
can be decomposed into 6 cases (In the last roll we arrive at $f(m-i)$, where $i \in \{1,2,3,4,5,6\}$).\\
Therefore, $1-f(m)=\sum_{i=1}^{6}\frac{6-i}{6}f(m-i)$. Suppose $p_{0}=\{f(m)\}_{m=1}^{\infty}$,
we get 
\begin{equation}
    1-p_{0}=\sum_{i=1}^{6}\frac{6-i}{6}p_{0}
\end{equation}

Solving Eq.(7), we get $p_{0}=\frac{2}{7}$.\\
\\
(iii) (This is one student's solution) \\
Let $a_{m}=\max _{m-5 \leqslant n \leqslant m} f(n)$ and $b_{m}=\min _{m-5 \leqslant n \leqslant m} f(n)$,
we will prove that $a_{m}$ is non-increasing and $b_{m}$ is non-decreasing.\\
$f(m+1)=\frac{1}{6} \sum_{n=m-5}^{m} f(n) \leqslant \max _{m-5 \leqslant n \leqslant m} f(n)=a_{m}$,\\
$a_{m+1}=\max _{m-4 \leqslant n \leqslant m+1} f(n) \leqslant \max \{f(m+1), a_{m}\}=a_{m}$.\\
Similarly, we can prove that $b_{m+1} \ge b_{m}$.\\
Since $0 \leqslant b_{m} \leqslant b_{m+1} \leqslant a_{m+1} \leqslant a_{m} \leqslant 1$, both $\{a_{m}\}$ and $\{b_{m}\}$ converge.\\

Let $\lim_{m \to \infty}a_{m}=a$ and $\lim_{m \to \infty}b_{m}=b$, we will prove that $a=b$.\\
If $a < b$, then there extists some $m_{0}$ that $a_{m_{0}} < b_{m_{0}}$, contradict.\\
If $a > b$, let $\epsilon < \frac{1}{12}(a-b)$, $\exists M\in\mathbb{N}, \forall m \geq M, a-\epsilon<a_{m}<a+\epsilon$
and $b-\epsilon<b_{m}<b+\epsilon$.\\
Thus $\forall m \geq M+6, f(m) \leq \frac{5}{6}(a+\epsilon)+\frac{1}{6}(b+\epsilon)\leq a-\epsilon$, 
which means $a_{m} \leq a-\epsilon$, contradict.\\
So $a=b$ and $f(m)$ converges.$\Box$ 
\end{document}
