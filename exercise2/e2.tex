\documentclass{article}
\usepackage{amsmath}
\usepackage{amssymb}
\usepackage[thehwcnt=2]{iidef}
\DeclareMathOperator{\bP}{\mathbb{P}}
\DeclareMathOperator{\Bern}{Bern}
\DeclareMathOperator{\mF}{\mathcal{F}}
\usepackage[utf8]{inputenc}
\thecourseinstitute{Tsinghua-Berkeley Shenzhen Institute}
%\title{ldp-1}
%\author{zhaof17 }
%\date{March 2021}
\thecoursename{Probability}
\theterm{Spring 2021}
\setlist[enumerate,2]{label=(\roman*)}
\begin{document}

\courseheader
\name{Feng Zhao}

\begin{enumerate}
\item Let $X_1,X_2,X_3$ represent the
random variables of throwing dices for each term. They are independent. The event that
we obtain 3 numbers in strictly increasing
order is equivalent with $X_1<X_2 \wedge X_2 <X_3$.
\begin{align*}
    P(X_1<X_2 \wedge X_2 <X_3)
    &=\sum_{i=2}^5 P(X_1 < i \wedge X_3 > i)P(X_2=i)\\
    &=\sum_{i=2}^5 P(X_1 < i)P( X_3 > i)P(X_2=i) \\
    &=\sum_{i=2}^5  \frac{1}{6}\frac{i-1}{6}\frac{6-i}{6}\\
    &=\frac{5}{54}
\end{align*}
Similarly, $P(X_1>X_2 \wedge X_2 > X_3)=\frac{5}{9}$.
\item 
\begin{enumerate}
    \item Using the log sum inequality we have
$\sum_{i=1}^n p_i\log \frac{p_i}{C r^u} \geq 0$,
which is equivalent to
$\sum_{i=1}^n p_i\log p_i \geq \log C + \mu \log r = \sum_{i=1}^n Cr^{x_i} \log (Cr^{x_i})$, since $C, r$ satisfy the two
constraints $\sum_{i=1}^n Cr^{x_i}=1$
and$\sum_{i=1}^n Cr^{x_i}x_i=\mu$.
\item $p_i = Cr^i$, which has the same
PMF as the geometric distribution.
Let $p$ be the parameter of the
geometric distribution. Then
we have $\mu=\frac{1}{p}$,
$C=\frac{p}{1-p}=\frac{1}{\mu-1}$
and $r=1-p=1-\frac{1}{\mu}$.
\end{enumerate}
\item
\begin{enumerate}
    \item Let the partial sum $T_{n}=\sum_{i=1}^n \frac{(-1)^{n+1}}{n}$.
    Also we define
    $S_{n}=\sum_{i=1}^n \frac{1}{n}$.
    Then we can show that
    $T_{2n}=S_{2n} - S_n$. Using the asymptotic form for harmonic series,
    we have $S_n \sim \log n + \gamma$ where
    $\gamma$ is the Euler constant.
    Therefore, $T_{2n} \sim \log (2n) - \log n \to = 2$. That is
    $\sum_{i=1}^{\infty} \frac{(-1)^{n+1}}{n}=\log 2$.
    \item Using similar method as above, we can
    show that
    $$
    1-\frac{1}{2}
    -\frac{1}{4}
    + \frac{1}{3}
    -\frac{1}{6}
    -\frac{1}{8} +\dots
    = \frac{1}{2}\log 2
    $$
\item $\sum_{i=1}^{\infty} \frac{1}{n}=\infty$.
\end{enumerate}
\item
\begin{enumerate}
    \item Using the equality
    $\mathbb{E}[(X-1)^k | X>1] = \mathbb{E}[X^k]$ 
    for geometric distribution can
    we get
    $\mathbb{E}[X^3] = \frac{p^2+6-6p}{p^3}$
    and $\mathbb{E}[X^4] = \frac{-p^3+14p^2-36p+24}{p^4}$.
    \item Using the equality
    $\mathbb{E}[X(X-1)\dots(X-k)]
    =\lambda^{k+1}$
    for Poisson distribution.
    We can get
    $\mathbb{E}[X^3]=\lambda^3+3\lambda^2+\lambda$
    and
    $\mathbb{E}[X^4]=\lambda^4+6\lambda^3+
    7\lambda^2+\lambda$
\end{enumerate}
\item $\beta_0=\frac{1}{2}$.
Let $A$ represents the event that no collision
happens. Then $P(A) = \frac{(n-1) \dots (n-k+1)}{n^{k-1}}$. We analyze the asymptotic behaviour of $P(A)$ by Stirling's formula as follows:
$P(A) = \frac{n!}{n^k (n-k)!} \sim \frac{(n/e)^n}{n^k ((n-k)/e)^{n-k}}$. We can write
$\frac{(n/e)^n}{n^k ((n-k)/e)^{n-k}}=\frac{1}{\exp(k+(n-k)\log(1-\frac{k}{n}))}=\frac{1}{\exp((n+k)k^2/(2n^2) + O(k^3/n^2)}$.
If $\beta < \frac{1}{2}$, $k^2 < n$ and $P(A) \to 1$.
Otherwise $P(A) \to 0$.
\item 
\begin{enumerate}
    \item The recursive formula is
    \begin{equation}\label{eq:fm}
    f(m) = \frac{1}{6}\sum_{i=1}^6 f(m-i)
    \end{equation}
    so that we get
    $f(2020)\approx 0.286$ by computer program.
    \item The complement of the event $\exists n, \textrm{s.t.} S_n=m$
    is $\exists n,\textrm{s.t.} S_n<m \wedge S_{n+1} > m$, which can
    be further decomposed into 5 events: $S_n = m-j+1 \wedge X_{n+1} \geq j$ for $j=2,3,4,5,6$. Therefore, we have the following equation:
    $1-f(m)=\sum_{j=1}^5 \frac{6-i}{6}f(m-j)$. Taking the limit $m\to \infty$
    on both sides we have $\lim_{m\to\infty}f(m)=\frac{2}{7}$.
    \item
    Let $f(0)=1$. Then $f(6)$ satisfies the recursive formula
    \eqref{eq:fm}.
        Since $0 \leq f(m)\leq 1$, then $\max_{m} |f(m-j)-f(m)| \leq 1$
        for $m\geq 0$ and $j=1,2,\dots, 5$. Using the recursive
        formula,
    first we have
    $|f(m)-f(m-j)|=\frac{1}{6}|\sum_{i=1}^6 [f(m-i)-f(m-j)]| \leq \frac{5}{6}$ for $m\geq 6$ and $j=1,2,\dots, 5$.
    Then we have $|f(m)-f(m-j)|=\frac{1}{6}|\sum_{i=1}^6 [f(m-i)-f(m-j)]| \leq (\frac{5}{6})^2$ for $m\geq 12$ and $j=1,2,\dots, 5$.
    Recursively we have
    $[f(m)-f(m-j)]| \leq (\frac{5}{6})^{\lfloor m/6 \rfloor} $ for $j=1,2,\dots, 5$.
    Then
    $|f(m+n)-f(m)|\leq 
    \sum_{i=1}^n |f(m+i)-f(m+i-1)| \leq \sum_{i=1}^n (\frac{5}{6})^{\lfloor (m+i)/6 \rfloor} \leq
    6 \frac{1}{1-(5/6)} (\frac{5}{6})^{\lfloor (m+1)/6 \rfloor}=36 (\frac{5}{6})^{\lfloor (m+1)/6 \rfloor}$.
    By Cauchy's convergence test, $f(m)$ converges.
\end{enumerate}
\end{enumerate}

\end{document}

