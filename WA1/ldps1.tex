\documentclass[a4paper, 12pt, answers]{exam}
\usepackage[T1]{fontenc}
\usepackage{amsmath}
\usepackage{amssymb}
\usepackage{enumerate}
\usepackage{bm}
\usepackage{advdate}
\usepackage{datetime}
\usepackage[mathcal]{eucal}
\usepackage{dsfont}
\usepackage[numbered,framed]{matlab-prettifier}
\usepackage{tkz-euclide}
\usetkzobj{all}
\usepackage{url}
\usepackage{hyperref}
\usepackage{verbatim}
\hypersetup{
  colorlinks   = true, %Colours links instead of ugly boxes
  urlcolor     = blue, %Colour for external hyperlinks
  linkcolor    = blue, %Colour of internal links
  citecolor   = red %Colour of citations
}

\allowdisplaybreaks
\newdate{issuedate}{22}{2}{2020}
\newdate{duedate}{7}{3}{2020}

% \newcommand{\duedate}[1][14]{%
% \begingroup
% \AdvanceDate[#1]%
% \today%
% \endgroup
% }%
%% Definitions for Learning from Data
%% UPDATED: November 20, 2018 by Xiangxiang 
\newcommand{\theterm}{Spring 2020}

\newcommand{\thecoursename}{
Tsinghua-Berkeley Shenzhen Institute\\
%\vspace*{0.1in}
\textsc{Introduction to Probability Theory}
}

\newcommand{\courseheader}{
\vspace*{-1in}
\begin{center}
\thecoursename \\
\theterm
\vspace*{0.1in}
\hrule
\end{center}
}
\newcommand{\uc}{\underline{c}}    % c, vec
\newcommand{\uv}{\underline{v}}    % x, vec
\newcommand{\uw}{\underline{w}}    % w, vec
\newcommand{\ux}{\underline{x}}    % x, vec
\newcommand{\uy}{\underline{y}}    % y, vec
\newcommand{\uz}{\underline{z}}    % z, vec
\newcommand{\um}{\underline{m}}    % m, vec
\newcommand{\ut}{\underline{t}}    % t, vec

\newcommand{\bA}{\bm{A}}    % A, mat
\newcommand{\bn}{\bm{n}}    % n, mat

\newcommand{\bc}{\bm{c}}    % c, vec
\newcommand{\bu}{\bm{u}}    % u, vec
\newcommand{\bv}{\bm{v}}    % v, vec
\newcommand{\bw}{\bm{w}}    % w, vec
\newcommand{\bT}{\bm{T}}    % T, mat

\newcommand{\bx}{\bm{x}}    % x, vec
\newcommand{\bX}{\bm{X}}    % X, mat
\newcommand{\by}{\bm{y}}
\newcommand{\bY}{\bm{Y}}     % y, vec
\newcommand{\rvby}{\bm{\mathsf{y}}}    % y, rv. vec
\newcommand{\rvbx}{\bm{\mathsf{x}}}    % x, rv. vec
\newcommand{\bz}{\bm{z}}    % z, vec
% \newcommand{\bm}{\bm{m}}    % m, vec
\newcommand{\bt}{\bm{t}}    % t, vec
\newcommand{\bzero}{\bm{0}}    % 0, vec

\newcommand{\balpha}{\bm{\alpha}}    % alpha, vec
\newcommand{\bxi}{\bm{\xi}}    % xi, vec
\newcommand{\btheta}{\bm{\theta}}
\newcommand{\bTheta}{\bm{\Theta}}    % theta, vec
\newcommand{\bmu}{\bm{\mu}}    % mu, vec

\newcommand{\bSigma}{\bm{\Sigma}}    % Sigma, vec

\newcommand{\cL}{\mathcal{L}}  
\newcommand{\cX}{\mathcal{X}}  


\newcommand{\rvx}{\mathsf{x}}    % x, r.v.
\newcommand{\rvy}{\mathsf{y}}    % y, r.v.
\newcommand{\rvz}{\mathsf{z}}    % z, r.v.
\newcommand{\rvw}{\mathsf{w}}    % w, r.v.
\newcommand{\rvv}{\mathsf{v}}    % v, r.v.
\newcommand{\rvm}{\mathsf{m}}    % m, r.v.
\newcommand{\rvt}{\mathsf{t}}    % t, r.v.
\newcommand{\rvH}{\mathsf{H}}    % H, r.v.
\newcommand{\urvx}{\underline{\mathsf{x}}}    % x, r.v. vec
\newcommand{\urvy}{\underline{\mathsf{y}}}    % y, r.v. vec
\newcommand{\urvz}{\underline{\mathsf{z}}}    % z, r.v. vec
\newcommand{\urvw}{\underline{\mathsf{w}}}    % w, r.v. vec
\newcommand{\urvt}{\underline{\mathsf{t}}}    % t, r.v. vec
\newcommand{\defeq}{\triangleq} %\coloneqq
\newcommand{\reals}{\mathbb{R}}
\newcommand{\T}{\mathrm{T}}    % transpose
\newcommand{\F}{\mathrm{F}}    % Frobenius
\newcommand{\BLS}{\mathrm{BLS}}    % BLS
\newcommand{\LLS}{\mathrm{LLS}}    % LLS
\newcommand{\MVU}{\mathrm{MVU}}    % MVU

\DeclareMathOperator*{\maximize}{maximize}    % maximize
\DeclareMathOperator*{\minimize}{minimize}    % minimize
\newcommand{\st}{\mathrm{subject~to}}    % minimize

% \newcommand{\E}[1]{\mathbb{E}\left[{#1}\right]}
% \newcommand{\Prob}[1]{\mathbb{P}\left({#1}\right)}
\DeclareMathOperator*{\argmax}{arg\,max}
\DeclareMathOperator*{\argmin}{arg\,min}
\DeclareMathOperator*{\argsup}{arg\,sup}
\DeclareMathOperator*{\arginf}{arg\,inf}
\DeclareMathOperator{\tr}{tr}
\DeclareMathOperator{\Var}{Var}
\DeclareMathOperator{\Cov}{Cov}
\DeclareMathOperator{\MSE}{MSE}
\DeclareMathOperator{\1}{\mathds{1}}
\DeclareMathOperator{\E}{\mathbb{E}}
\DeclareMathOperator{\Prob}{\mathbb{P}}
\DeclareMathOperator{\im}{im}
\DeclareMathOperator{\rank}{rank}

\newcommand\independent{\protect\mathpalette{\protect\independenT}{\perp}}
\def\independenT#1#2{\mathrel{\rlap{$#1#2$}\mkern2mu{#1#2}}}


\makeatletter
\@ifclasswith{exam}{answers}{\newcommand{\firstblock}{comments}}{}
\makeatother

\begin{document}

\pagestyle{headandfoot}
\runningheadrule


\newcounter{psctr}
\setcounter{psctr}{1} % set to the times of problem

\runningheader{Homework Assignment \thepsctr}
              {\textsc{Introduction to Probability Theory}}
              { Page \thepage\ of \numpages}
\firstpagefooter{}{}{}
\runningfooter{}{}{}


\newcounter{Sequ}
\newenvironment{Sequation}
   {\stepcounter{Sequ}%
     \addtocounter{equation}{-1}%
     \renewcommand\theequation{S\arabic{Sequ}}\equation}
   {\endequation}
%\topskip0pt

% \vspace*{\fill}
\centering

% \vspace{0.3em}
\centering
\renewcommand{\thequestion}{\arabic{psctr}.\arabic{question}}
\courseheader

\begin{center}
  \underline{\bf Written Assignment \thepsctr} \\
\end{center}
\begin{flushleft}
  \textbf{Issued:} \displaydate{issuedate} \hfill
  \textbf{Due:} \displaydate{duedate} 
\end{flushleft}

\hrule 

\ifdefined\firstblock
\input{\firstblock}
\fi

%\pointname{}
%\vspace{\footskip}
%\vspace{1em}

% \begin{flushleft}
%   Consider the problem of classifying $l$ samples $(\bx_i, y_i)$ using SVM, where $\bx_i \in \mathbb{R}^n$, $y_i \in \{-1, 1\}$, $(i = 1, \cdots, l)$.
% %{\bf Notations:} 
% \end{flushleft}


\begin{questions}
  
  

  \question 
  \begin{parts}
  	\part Show that the union of countably many countable sets is countable.
  	\part A real number $x$ is rational if $x = m/n$, where $m$ is an integer and $n$ is a nonzero integer. Show that the set of rational numbers $\mathbb{Q}$ is countable
  \end{parts}
\begin{solution}
	\begin{parts}
	\part	Denote the sets as $A_i (i \in \mathbb{N}(\textrm{including 0}))$ and the elements of $A_i$ as $a_{ij} (j \in \mathbb{N})$. For the most general case, which means disjoint countably infinite countable infinite sets, we need to find a bijection $f: \mathbb{N} \times \mathbb{N} \rightarrow \mathbb{N}$.\\
	For instance, $f(i,j)=2^i(2j+1)-1$ or $f(i,j)=\frac{(i+j)(i+j+1)}{2}+j$.
	\part Let $A_i=\{\frac{0}{i}, \frac{1}{i}, \dots\}$, $B_i=\{-\frac{0}{i}, -\frac{1}{i}, \dots\}$. According to (a), $(\cup_i A_i)\cup (\cup_i B_i) = \mathbb{Q}$ is countable.
	\end{parts}
\end{solution}
 \question
 Let $\{x_n\}$ and $\{y_n\}$ be real sequences that converge to $x$ and $y$, respectively. Provide a formal proof of the fact that $x_ny_n$ converges to $xy$.
 \begin{solution}
 Fix some $\epsilon > 0$. Let $M=\max\{|x_n|\}$. Let $n_1$ be such that $|x_n-x| < \frac{\epsilon}{2y}$, for all $n>n_1$. Let $n_2$ be such that $|y_n-y| < \frac{\epsilon}{2M}$, for all $n>n_2$. Let $n_0 =\max\{n_1, n_2\}$. Then, for all $n>n_0$, we have
\begin{equation*}
|x_n y_n-xy| =|x_n(y_n-y)+(x_n-x)y|\le |x_n(y_n-y)|+ |(x_n-x)y| \le \frac{\epsilon}{2M}M+\frac{\epsilon}{2y}y= \epsilon
\end{equation*}
which proves the desired result.
 \end{solution}
 \question
 Let $\Omega=\mathbb{N}$ (the positive integers), and let $\mathcal{F}_0$ be the collection of subsets of $\Omega$ that either have finite cardinality or their complement has finite cardinality. For any $A \in \mathcal{F}_0$, let $\Prob(A)=0$ if $A$ is finite, and $\Prob(A)=1$ if $A^c$ is finite.
 \begin{parts}
 	\part Show that $\mathcal{F}_0$ is a field but not a $\sigma$-field.
 	\part Show that $\Prob$ is finitely additive on $\mathcal{F}_0$; that is, if $A,B \in \mathcal{F}_0$, and $A,B$ are disjoint, then $\Prob(A \cup B)= \Prob(A)+ \Prob(B)$.
 	\part Show that $\Prob$ is not countably additive on $\mathcal{F}_0$; that is, construct a sequence of
 	disjoint sets $A_i \in \mathcal{F}_0$ such that $\cup_{i=1}^{\infty}A_i \in \mathcal{F}_0$ and $\Prob(\cup_{i=1}^{\infty}A_i) \neq \sum_{i=1}^{\infty} \Prob(A_i)$.
 	\part Construct a decreasing sequence of sets $A_i \in \mathcal{F}_0$ such that $\cap_{i=1}^{\infty}A_i=\varnothing$ for which $\lim\limits_{i \rightarrow \infty} \Prob(A_i) \neq 0$.
 \end{parts}
\begin{solution}
\begin{parts}
	\part The empty set has zero cardinality, and therefore belongs to $\mathcal{F}_0$. Furthermore, if $A \in \mathcal{F}_0$, then either $A$ or $A^c$ has finite cardinality. It follows that either $A^c$ or $(A^c)^c$ has finite cardinality, so that $A^c \in \mathcal{F}_0$. \\
	Suppose that $A, B \in \mathcal{F}_0$. If both $A$ and $B$ are finite, then $A \cup B$ is also finite and belongs to $\mathcal{F}_0$. Suppose now that at least one of $A$ or $B$ is infinite. We have $A \cup B =(A^c \cap B^c)^c$. Since $A^c \cap B^c$ is finite, it follows that $A \cup B \in \mathcal{F}_0$. This shows that $\mathcal{F}_0$ is a field.\\
    To see that $\mathcal{F}_0$ is not a $\sigma$-field, note that $\{2n\} \in \mathcal{F}_0$ for every $n \in \mathbb{N}$, but the set $\cup_{n=0}^{\infty} {2n}$, the set of even natural numbers, is not in $\mathcal{F}_0$.
    \part Let $A, B \in \mathcal{F}_0$ be disjoint. If both $A$ and $B$ are finite, then $ \Prob(A \cup B)=0=\Prob(A)+\Prob(B)$. Suppose that either $A$ or $B$ (or both) is infinite. Since $A$ and $B$ are disjoint, we have $A \subset B^c$ and $B\subset A^c$. It follows that $A$ and $B$ cannot both be infinite. Therefore, $\Prob(A \cup B)=1=\Prob(A)+\Prob(B)$, and $\Prob$ is finitely additive.
    \part Note that $\{i\} \in \mathcal{F}_0$ and $\cup_{i \ge 1}\{i\}=\Omega$. However, $\Prob(\{i\})=0$ while $\Prob(\Omega)=1$, hence $\Prob$ is not countably additive.
    \part Let $A_i =\{i, i+1, \dots\}$. Then $(A_i)_{i\ge1}$ forms a decreasing sequence of sets with $\cap_i A_i =\varnothing$. But $\Prob(A_i)=1$ for all n, hence $\lim\limits_{i \rightarrow \infty} \Prob(A_i)=1$.
\end{parts}
\end{solution}
\end{questions}  


\end{document}
%%% Local Variables:
%%% mode: latex
%%% TeX-master: t
%%% End:

%  LocalWords:  headandfoot covariance
