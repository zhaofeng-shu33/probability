\documentclass{article}
\usepackage{amsmath}

\title{random walk and gambling problem}
\author{zhaof17 }
\date{May 2021}

\begin{document}
\maketitle

We consider 1-dimensional asymmetric random walk problem, in which $P(X_i=1)=p$
and $P(X_i=-1)=1-p=q$.
We consider the probability of event $A_j$ in which the particle
hits $i=0$ before it hits $i=N$ starting from position $i=j$ for $j=0,1,\dots, N$.
Let $h_j=P(A_j)$.
Clearly, $h_0=1,h_N=0$.
For other $0<j<N$ it satisfies the following
equation: $h_j = qh_{j-1} + ph_{j+1}$.
It is a linear equation system specified by the following Toeplitz tri-diagonal matrix, which has unique solution.
\begin{equation}
    \begin{pmatrix}
    1 & -p & 0 & \dots & 0\\
    -q & 1 & -p & \dots & 0\\
    0 & -q & 1 & -p & \dots \\
    \vdots & & \vdots& & \vdots \\
    0 & \dots &  0 & -q & 1 
    \end{pmatrix}\begin{pmatrix}
    h_1 \\ h_2 \\ h_3 \\ \vdots \\ h_{N-1}
    \end{pmatrix} = \begin{pmatrix}
    q \\ 0 \\ 0 \\ \vdots \\ 0
    \end{pmatrix} 
\end{equation}
The solution is 
\begin{equation}\label{eq:hi}
h_i = 
\begin{cases}
\frac{(q/p)^i - (q/p)^N}{1-(q/p)^{N}} & p\neq q\\
\frac{N-i}{N} & p=q=\frac{1}{2}
\end{cases}    
\end{equation}
for $i=0, 2, \dots, N$
.
When $p=q=\frac{1}{2}$, we can represent
$h_2, h_3, \dots$ supposing $h_1$
is known. Then $h_2=2h_1-1, h_3=3h_1-2,
\dots, h_i=i h_1 - (i-1)$. Using the last
equation $\frac{h_{N-2}}{2}
=h_{N-1}$ we can solve out $h_1=\frac{N-1}{N}$
and $h_i=\frac{N-i}{N}$
as required.

When $p\neq q$, we consider the homogeneous linear
equation, $h_{j+1}=\frac{h_j - q h_{j-1}}{p}$
for $j=1,2,\dots, N-1$.
The characteristic function is given by
$px^2=x-q$, which gives two roots $x=1$
and $x=\frac{q}{p}$.
Therefore, the general formula
is given by $h_j=a+b(q/p)^j$. The initial
condition is $h_0=1$ and the end condition
is $h_N = 0$ so that we can solve out $a,b$
and get the result in \eqref{eq:hi}.

The above model is equivalent to a gambling problem with two players. At the beginning,
player 1 has $i$ units of fortune while
player 2 has $N-i$ units such that their total
fortune is $N$. In each turn, player 1 wins
with probability $p$ while player 2 wins
with probability $q$. We are concerned with
the event that player 1 loses all its fortune first. Such probability is also given by
\eqref{eq:hi}.
\end{document}

