\documentclass{article}
\usepackage{amsmath}
\usepackage{amssymb}
\usepackage[thehwcnt=5]{iidef}
\DeclareMathOperator{\bP}{\mathbb{P}}
\DeclareMathOperator{\Bern}{Bern}
%\DeclareMathOperator{\E}{\mathbb{E}}
\DeclareMathOperator{\mF}{\mathcal{F}}
\usepackage[utf8]{inputenc}
\thecourseinstitute{Tsinghua-Berkeley Shenzhen Institute}
%\title{ldp-1}
%\author{zhaof17 }
%\date{March 2021}
\thecoursename{Probability}
\theterm{Spring 2021}
%\setlist[enumerate,2]{label=(\roman*)}
\begin{document}

\courseheader
\name{Feng Zhao}

\begin{enumerate}
\item The answer is yes. $X+Y$ may not be Gaussian
distribution. 
Let $X\sim \mathcal{N}(0,1)$ and $Y=(-1)^Z X$ where $Z \sim \Bern(\frac{1}{2})$ and $Z$ is independent with $X$. We can verify $Y$ follows standard normal
distribution by the symmetric property of
normal distribution. However, $P(X+Y=0)=\frac{1}{2}$, which implies that
$X+Y$ can not be Gaussian distribution.

\item $X \sim \mathcal{N}(0,1)$, then $Z=(X+Y)/\sqrt{2}$ has the same distribution as
$X$. Let $\varphi(t)$ be the common characteristic
function of $X$ and $Z$. Since $\varphi(t)=
\E[\exp(it\frac{X+Y}{\sqrt{2}})]
=\E[\exp(it\frac{X}{\sqrt{2}})]
\E[\exp(it\frac{Y}{\sqrt{2}})]=
\varphi^2(\frac{t}{\sqrt{2}})$.
Then we have the identity $\varphi(t)=\varphi^2(\frac{t}{\sqrt{2}})$.
Let $f(t) = \ln \varphi(t)$. Then $f(t)$
satisfies $f(t)=2f(\frac{t}{\sqrt{2}})$.
$f(0)=\ln \varphi(0)=0$.
Taking the derivative on both sides and let
$t=0$, we get $f'(0)=0$. Recursively applying
the formula, we get $f(t) = 2^{2n} f(\frac{t}{2^{n}})$. For fixed $t$,
we expand $f(\frac{t}{2^{n}})$ at zero:
$f(t) = \frac{1}{2}f''(0)t^2 + \frac{1}{6}f^{(3)}(\xi)\frac{t^3}{2^n}$
where $\xi \in (0, \frac{t}{2^{n}})$.
Let $n\to \infty$, we have $f(t)=\frac{1}{2}f''(0)t^2$.
Therefore, $\varphi(t)=\exp(Ct^2)$ for some constant $C$.
Since $\E[X^2]=-\frac{1}{2}\varphi''(t)|_{t=0}=1 \implies C=-\frac{1}{2}$. The characteristic function is indeed
that of the standard normal distribution. By the uniqueness
of the characteristic function, we conclude that $X$
follows standard normal distribution.

\item we can write the pdf of $(X_1, \dots, X_{2n-1})$
as 
$$
f_{X_1, \dots, X_{2n-1}}(x_1, \dots, x_{2n-1})
= c_n \exp(-\frac{1}{2} x^T A x)
$$
By the definition of $f$, $A$ is a positive 
definite matrix. Therefore, $(X_1, \dots, X_{2n-1})$
follow jointly Gaussian distributions. The determinant of $A$ is:
\begin{align*}
    |A|=|\begin{pmatrix}
    1 & -1 & 0 & \dots \\
    -1 & 2 & 1 & \dots \\
    \vdots & \vdots & \vdots & \vdots \\
    \dots & -1 & 2 & -1 \\
    \dots & 0 & -1 & 2
    \end{pmatrix}|
    =|\begin{pmatrix}
    1 & -1 & 0 & \dots \\
    0 & 1 & 1 & \dots \\
    \vdots & \vdots & \vdots & \vdots \\
    \dots & 0 & 1 & -1 \\
    \dots & 0 & 0 & 1
    \end{pmatrix}| = 1
\end{align*}
Therefore $c_n = (2\pi)^{1/2 - n}$. Besides,
$\Var[X_n] = 2$.
\item
\begin{enumerate}[label=(\roman*)]
    \item The answer is yes. $\{S_n / n \}_{n=1}^{\infty}$
    converge in distribution to Cauchy distribution
    with PDF $f(x)=\frac{1}{\pi (1+x^2)}$.
    The characteristic function of $S_n$
    is $\varphi(t)=\exp(-n|t|)$. Then the characteristic
    function of $S_n/n$ is $\varphi(t)=\exp(-|t|)$.
    Therefore, $S_n/n$ follows Cauchy distribution with
    parameter $1$.
    \item The answer is yes. $\{S_n / n^2 \}_{n=1}^{\infty}$
    converge in distribution to $0$. The characteristic
    function of $S_n/n$ is $\varphi(t)=\exp(-|t|/n)$, which
    converges to 1 for all given $t$. The constant $1$
    is the characteristic function of random variable 0.
    By Lévy's continuity theorem, we have the aforementioned claim.
    \item The answer is no. Since Cauchy distribution
    does not have finite mean, Central Limit Theorem
    cannot be applied. $\{S_n / \sqrt{n} \}_{n=1}^{\infty}$
    does not converge in distribution.
\end{enumerate}
\item By Taylor
expansion, we have $-\frac{1}{2} x^2-\frac{1}{12}  x^4 \leq \log \cos x \leq -\frac{1}{2} x^2$ for $|x|<0.5$.
We claim that $\beta_0 = \sqrt{2}$. The
characteristic function $X_j$ is $\varphi_j(t)=\E[\exp(iX_j t)]
=\frac{e^{itj}+e^{-itj}}{2}=\cos (j t)$. The the characteristic
function for $S_n$ is $\phi_n(t)=\prod_{j=1}^{n} \cos(jt)$.
\begin{enumerate}
    \item $\{S_n / n^2 \}_{n=1}^{\infty}$
    converge to $0$ in distribution. To prove that
    we need only to show that
    $\lim_{n\to \infty}\phi_n(\frac{t}{n^2}) \to 1$ for any given $t$, which is equivalent to show
    that
    $\lim_{n\to \infty} \sum_{j=1}^{n} \log \cos(\frac{jt}{n^2}) = 0$. It is obvious that
    $\sum_{j=1}^{n} \log \cos(\frac{jt}{n^2}) \leq 0$.
    On the other hand, we use Taylor expansion:
    $\sum_{j=1}^{n} \log \cos(\frac{jt}{n^2})
    \geq -\frac{1}{2}\sum_{j=1}^{n} \frac{j^2t^2}{n^4}
    -\frac{1}{12}\sum_{j=1}^{n} \frac{j^4t^4}{n^8}=O(\frac{1}{n}) \to 0$.
    \item $\{S_n / n^{3/2} \}_{n=1}^{\infty}$
    converge to $\mathcal{N}(0, \frac{1}{3})$ in distribution.
    We use similar method with (a) to show this fact.
    We only need to show that $\lim_{n\to \infty}\phi_n(\frac{t}{n^{3/2}}) \to \exp(-\frac{1}{6}t^2)$,
    which is further equivalent with
    $\lim_{n\to \infty} \sum_{j=1}^{n} \log \cos(\frac{jt}{n^{3/2}}) = -\frac{1}{6}t^2$.
    Notice that
    \begin{align*}
     -\frac{t^2}{2n^3}\sum_{j=1}^n j^2 -\frac{t^4}{12n^6}\sum_{j=1}^n j^4   \leq \sum_{j=1}^{n} \log \cos(\frac{jt}{n^{3/2}})
        \leq -\frac{t^2}{2n^3}\sum_{j=1}^n j^2
    \end{align*}
    Since $\frac{1}{n^6}\sum_{j=1}^n j^4 = O(\frac{1}{n})$
    while $\frac{1}{n^3}\sum_{j=1}^n \sim \frac{1}{3}$,
    we reach
    $\lim_{n\to \infty} \sum_{j=1}^{n} \log \cos(\frac{jt}{n^{3/2}}) = -\frac{1}{6}t^2$.
    \item Let $\Phi$ be the CDF of Gaussian random variable
    with variance $\frac{1}{3}$.
    $\lim_{n\to \infty} P(\frac{S_n}{n} \leq x)
    =\lim_{n\to \infty} P(\frac{S_n}{n^{3/2}} \leq \frac{x}{\sqrt{n}}) = \Phi(0) = \frac{1}{2}$.
\end{enumerate}

\item 
\begin{enumerate}
    \item Yes, $\{S_n\}_{n=1}^{\infty}$ converges almost
    surely to some random variable. For any $w\in \Omega$,
    $S_n(w)$ is an increasing sequence. Therefore,
    the event that $\{S_n\}_{n=1}^{\infty}$ does not converge
    is equivalent to $S_n = \infty$.
    $P(S_n = \infty) \leq P(X_i=1 \textrm{ for infinite } i) = 0$. Since $S_n$ are random variables, $\lim_{n} S_n$ is also a random variable.
    $\E[Y_n]=\frac{1}{2^n}\sum_{i=1}^n \frac{1}{2^i}\binom{n}{i}=(\frac{3}{4})^n$. Then
    $\E[S_n]=\sum_{i=1}^n \E[Y_i] =\sum_{i=1}^n (\frac{3}{4})^i \implies \E[S]=\sum_{i=1}^{\infty} (\frac{3}{4})^i=3$ where $S=\lim_{n\to\infty} S_n$.
    To calculate $\Var[S]$, first we have
    $\E[Y_nY_{n+m}]=\E[X^2_1]^n[X_1]^m=(\frac{5}{8})^n(\frac{3}{4})^m$. Then $\E[S_n^2] = \sum_{i=1}^n (\frac{5}{8})^i
    + 2 \sum_{i=1}^n (\frac{5}{8})^i \sum_{j=i+1}^n
    (\frac{3}{4})^{j-i}=7\sum_{i=1}^n (\frac{5}{8})^i-6(\frac{3}{4})^{n}\sum_{i=1}^n (\frac{5}{6})^i$. Finally $\Var[S]=\E[S^2]-\E[S]^2=
    \lim_{n\to \infty} \E[S_n^2] - 9 = 
   7\sum_{i=1}^{\infty} (\frac{5}{8})^i-9=\frac{8}{3} $
   \item By the same argument as above, in such case
   $\{S_n\}_{n=1}^{\infty}$ also converge almost surely to some limit random variable.
\end{enumerate}
\end{enumerate}

\end{document}

