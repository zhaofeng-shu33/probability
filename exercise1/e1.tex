\documentclass{article}
\usepackage{amsmath}
\usepackage{amssymb}
\usepackage[thehwcnt=1]{iidef}
\DeclareMathOperator{\bP}{\mathbb{P}}
\DeclareMathOperator{\Bern}{Bern}
\DeclareMathOperator{\mF}{\mathcal{F}}
\usepackage[utf8]{inputenc}
\thecourseinstitute{Tsinghua-Berkeley Shenzhen Institute}
%\title{ldp-1}
%\author{zhaof17 }
%\date{March 2021}
\thecoursename{Probability}
\theterm{Spring 2021}
\begin{document}

\courseheader
\name{Feng Zhao}

\begin{enumerate}
\item This does not guarantee that
$X,Y,Z$ are mutually independent.
Counter example:
$$
P(x,y,z)= \frac{1}{8} (1+\epsilon\cdot (-1)^{x+y+z}), \textrm{ where } x,y,z \in {0,1}
$$
$\epsilon$ is chosen smaller enough to make
$P(x,y,z) > 0$ for every triple $(x,y,z)$.
The construction is a valid discrete probability measure.
We can verify that $X,Y,Z \sim \Bern(\frac{1}{2})$ and the pairwise independence holds. However, the
collective (mutual) independence does not hold, since $P(x=1,y=1,z=1)\neq P(x=1)\cdot
P(y=1)\cdot P(z=1)$.
\item $P(A_1)=\frac{1}{128}$, $P(A_2)=\frac{1}{64}$,
$P(A_3)=\frac{\binom{8}{4}}{256}=\frac{35}{128}$, $P(A_4)=\frac{1}{32}$, $P(A_1| A_3)
=\frac{P(A_1)}{P(A_3)}=\frac{1}{35}$,
$P(A_2|A_3)=\frac{P(A_2)}{P(A_3)}=\frac{2}{35}$.
\item
\begin{enumerate}
    \item Let $B_i$ represents the event that exactly $i$ corners
    of the cube are colored red.
    Then $P(A|B_5) = \frac{24}{56}=\frac{3}{7}$.
    \item Let $B=\bigcup_{i=5}^8 B_i$, then $P(A|B)=\frac{\sum_{i=5}^8 P(A\cap B_i)}{\sum_{i=5}^8 P(B_i)} =\frac{24p^5(1-p)^3 + 24 p^6(1-p)^2  + 8 p^7(1-p) + p^8}{56p^5(1-p)^3 + 28 p^6(1-p)^2 + 8 p^7(1-p) + p^8}$
\item Using Bayes rules, we can write $P(A)=P(A|B)P(B)+
+P(A|B_4)P(B_4)=6p^4(1-p)^4+24p^5(1-p)^3 + 24 p^6(1-p)^2  + 8 p^7(1-p) + p^8$
\end{enumerate}
\item
\begin{enumerate}
    \item $\sigma(C)=16$. Notice that 
    $\Omega=\{HH,HT,TT,TH\}$ which is equivalent with
    $\{1,2,3,4\}$. Further
    $A=\{HH,HT\}$, $B=\{HH,TH\}$, $\mathcal{C}=\{A,B\}$.
    Then $|\mathcal{C}| = |\sigma(\{ \{1\},\{2\},\{3\},\{4\}\})|
    =2^4=16$.
    \item Similarly, $|\mathcal{C}| = 32$.
\end{enumerate}
\item $\bP(\cap_{i=1}^{\infty} \cup_{n=i}^{\infty} A_n ) 
=1$. Proof: From the condition
$\bP(\cup_{n=1}^{\infty} A_n) = 1$ follows 
$\bP(\cap_{n=1}^{\infty} A^c_n) = 0$.
Since $\{A_n\}_{n=1}^{\infty}$
is an independent sequence,
we have $\prod_{n=1}^{\infty} P(A_n^c)=0$.
Since $\bP(A_n)<1$, $\bP(A_n^c)>0$. Then for any $i$,
$\prod_{n=i}^{\infty} P(A_n^c)=0 \Rightarrow \bP(\cap_{n=i}^{\infty} A_n^c) = 0$.
The probability
$\bP(\cup_{i=1}^{\infty}\cap_{n=i}^{\infty}A_n^c) 
\leq \sum_{i=1}^{\infty} \bP(\cap_{n=i}^{\infty}A_n^c)
= 0$. Taking the complement we get $\bP(\cap_{i=1}^{\infty} \cup_{n=i}^{\infty} A_n ) 
=1$ at last.
\item
\begin{enumerate}
    \item Since $\{w | X_1(w) + X_2(w) < c\}=\bigcup_{r\in Q}
    \{w|X_1(w) > r\} \cap \{w|X_2(w) < c-r\}$ is F-measurable,
    $X_1+X_2$ is a random variable.
    \item $\{w|\lim_{n\to \infty} X_n(w) \leq c\} = \bigcap_{n=1}^{\infty} \bigcup_{i=1}^{\infty} \bigcap_{j=i}^{\infty} \{w|X_n(w) < c + \frac{1}{n}\}$
    is F-measurable $\Rightarrow$ $\lim_{n\to\infty} X_n$
    is a random variable.
\end{enumerate}

\end{enumerate}

\end{document}
