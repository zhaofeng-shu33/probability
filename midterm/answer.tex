\documentclass{article}
\usepackage{amsmath}
\usepackage{enumitem}
\usepackage{amssymb}
\DeclareMathOperator{\bP}{\mathbb{P}}
\DeclareMathOperator{\E}{\mathbb{E}}
\DeclareMathOperator{\Exp}{Exp}
\DeclareMathOperator{\Cov}{Cov}
\DeclareMathOperator{\Var}{Var}
\newcommand\independent{\protect\mathpalette{\protect\independenT}{\perp}}
\def\independenT#1#2{\mathrel{\rlap{$#1#2$}\mkern2mu{#1#2}}}
%\title{ldp-1}
\author{Feng Zhao}
%\date{March 2021}
\title{Solution to Midterm exam}

\begin{document}

\maketitle
\begin{enumerate}
    \item omit
    \item omit
    \item $\E[X|X^2=k^2]
=kP[X=k|X^2=k^2]-kP[X=-k|X^2=k^2]
=k\frac{P[X=k]-P[X=-k]}{P[X=k]+P[X=-k]}$.
Therefore, we get $\E[X|X^2=0]=0,\E[X|X^2=1]=\frac{1}{2},
\E[X|X^2=4]=2,\E[X|X^2=9]=0$.
Combining the same value, we have $P(Y=0)=\frac{1}{3}$,
$P(Y=\frac{1}{2})=\frac{4}{9},P(Y=2)=\frac{2}{9}$.
\item
\begin{enumerate}
    \item Yes. 
    \begin{align*}
        \E[(S-\E[S])^3] &=
        \E[\sum_{i=1}^{10} (X_i - \E[X_i])^3]
        \\
        &=\sum_{i=1}^{10} \E[(X_i-\E[X_i])^3] + 6
        \sum_{1\leq i < j \leq 10}\E[(X_i-\E[X_i])^2(X_j-\E[X_j])]
        \\
        &+ 6
        \sum_{1\leq i < j <k \leq 10}\E[(X_i-\E[X_i])(X_j-\E[X_j])(X_k-\E[X_k])]\\
    \end{align*}
    Since $X_i \independent X_j$ for $i\neq j$,
    $\E[(X_i-\E[X_i])^2(X_j-\E[X_j])]=\E[(X_i-\E[X_i])^2]\E[(X_j-\E[X_j])]=0$
    and $\E[(X_i-\E[X_i])(X_j-\E[X_j])(X_k-\E[X_k])] = 0$.
    Therefore, $ \E[(S-\E[S])^3] = \sum_{i=1}^{10} \E[(X_i-\E[X_i])^3]$ holds.
    \item It is not true. Consider $X_i \sim N(0,1)$. Then $\E[X_i]=0$,
    $\E[X^4_i]=3$. Then the right hand side evaluates to 30 while
    the left hand side $\E[S^4]=300$.
\end{enumerate}
\item $Y_n(\omega)$ is monotonically increasing and bounded by 1.
Therefore, $Y_n \xrightarrow{a.s.} Y$. Heuristically we guess $Y$ is uniform distribution
over the interval $[0,1]$.
\item \begin{enumerate}
    \item $1 - (1,1)\begin{pmatrix}2&1\\1&2\end{pmatrix}^{-1}\binom{1}{1}= \frac{1}{3}$
    \item  We first compute the covariance matrix of $(X_1, X_2+X_3) $ as 
    $\begin{pmatrix}1&2\\2&6\end{pmatrix}$
    Then $1-\frac{4}{6}=
    \frac{1}{3}$
    \item Since $\Cov[X_1,X_2-X_3]=0$, $X_1 \independent X_2-X_3$
    and $\E[X_1|X_2-X_3]=\E[X_1]$ therefore, the value is $\Var[X_1]=1$.
\end{enumerate}
\item 
\begin{enumerate}[label=(\roman*)]
    \item Since $\E[X_i^2]=i^2\frac{1}{i\ln i}=\frac{i}{\ln i}$, $\E[Y_n]=\frac{1}{n^2}\sum_{i=2}^n \E[X_i^2]=\frac{1}{n^2}\sum_{i=2}^n \frac{i}{\ln i} \leq \frac{1}{n} \frac{n}{\ln n} \to 0$.
    Therefore, $\lim_{n\to\infty}\E[Y^2_n] = 0$.
    \item We first have $\E[Y_n]=0$. By Chebyshev's inequality,
    $P(|Y_n|>\epsilon)\leq \frac{\E[Y_n^2]}{\epsilon^2} \to 0$ as $n\to\infty$
    from (i).
    \item Let $A_n=\{\omega: |X_n(\omega)|=n\}$. Then $A = \limsup_{n\to\infty}A_n $. $P(A_n) = \frac{1}{n\log n}$,
    and $\sum_{n=1}^{+\infty} P(A_n) = \sum_{n=2}^{\infty}\frac{1}{n\log n}=+\infty$ by approximating it with $\int_{2}^{+\infty} \frac{1}{x\log x}dx$. Also $A_n$ are independent events. By Borel-Cantelli lemma, we have $P(A)=1$.
    \item For any $\omega$, if $Y_n(\omega)$  coverges,
    so is $X_n(\omega)$. But $X_n(\omega)$ contains
    infinitely items whose absolute value is $n$.
    That is, $X_n(\omega)$ does not converge. As a result,
    $X_n$ does not converge almost surely.
\end{enumerate}
\end{enumerate}

\end{document}

